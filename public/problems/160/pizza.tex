Rodrigo pediu uma pizza de mussarela de $N$ fatias, uma parte somente
com cebola e o resto somente com azeitonas. Entretanto, ao receber a
pizza em casa, notou que o motoqueiro que a entregou n�o foi cuidadoso
o suficiente, pois tanto as tiras de cebola quanto as azeitonas
estavam espalhadas por toda a pizza. Para piorar, como a pizza era de
mussarela, as tiras de cebola e as azeitonas estavam grudadas na
pizza. 

Como gosta mais de cebola do que de azeitona, Rodrigo deseja pegar
fatias consecutivas da pizza de tal forma que estas contenham a maior
diferen�a poss�vel entre tiras de cebola e azeitonas. Para isso, ele
contou quantas tiras e quantas azeitonas tinham em cada fatia e
subtraiu os dois valores, nessa ordem. Assim, sempre que uma fatia
contiver mais cebolas que azeitonas, ela recebe um n�mero positivo, e
caso contr�rio, um n�mero negativo. Uma fatia cujo n�mero seja zero
cont�m o mesmo n�mero de tiras de cebolas e azeitonas.

\begin{center}
  \epsfig{file=pizza.eps, scale=0.5} 
  
  Pizza
\end{center}

Por exemplo, supondo que as fatias contenham as seguintes diferen�as:
$5, -3, -3, 2, -1, 3$, pode-se pegar uma fatia consecutiva com $9$
cebolas a mais que azeitonas, utilizando as fatias com as diferen�as
$2, -1, 3, 5$ (lembre-se de que estamos tratando de um c�rculo e,
portanto, a fatia com diferen�a $5$ � vizinha da fatia com diferen�a
$3$ e vice-versa).

Como Rodrigo n�o entende de programa��o, ele resolveu contar com seus
servi�os.

\emph{OBS}: repare que � melhor n�o escolher nenhuma fatia caso
somente seja poss�vel escolher fatias consecutivas com mais azeitonas
que cebolas.

\section*{Tarefa}

Escreva um programa que, dados as diferen�as entre as quantidades de
cebolas e azeitonas em cada fatia de pizza, retorne a maior quantidade
poss�vel de cebolas que Rodrigo pode comer a mais do que a quantidade
de azeitonas utilizando somente fatias consecutivas de pizza.
(lembrando que a primeira fatia � adjacente � �ltima e vice-versa).

\section*{Entrada}

\inputnotice A primeira linha da entrada cont�m um inteiro $N$ que
indica o n�mero de fatias de pizza ($1 \leq N
\leq 100.000$). A segunda linha cont�m $N$ inteiros $K$ ($-100 \leq K
\leq 100$) separados por um espa�o em branco com as diferen�as
entre as quantidades de cebolas e de azeitonas.

\section*{Sa�da}

\outputnotice uma �nica linha, contendo a maior quantidade de cebolas
que Rodrigo pode comer a mais do que azeitonas.

\vspace{10pt}
\begin{center}
\begin{minipage}[c]{0.9\textwidth}
\begin{center}
\begin{tabular}{|l|l|l|} \hline
% Example 1
\begin{minipage}[t]{0.3\textwidth}
\bf{Entrada}
\begin{verbatim}
6
5 -3 -3 2 -1 3

\end{verbatim}
\bf{Sa�da}
\begin{verbatim}
9

\end{verbatim}
\end{minipage}
&
% Example 2
\begin{minipage}[t]{0.3\textwidth}
\bf{Entrada}
\begin{verbatim}
7
1 -2 2 -1 4 1 -5

\end{verbatim}
\bf{Sa�da}
\begin{verbatim}
6

\end{verbatim}
\end{minipage}
&
% Example 3
\begin{minipage}[t]{0.3\textwidth}
\bf{Entrada}
\begin{verbatim}
2
-3 -10

\end{verbatim}
\bf{Sa�da}
\begin{verbatim}
0

\end{verbatim}
\end{minipage}\\
\hline
\end{tabular}
\end{center}
\end{minipage}
\end{center}